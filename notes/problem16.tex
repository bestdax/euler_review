\chapter{Power Digit Sum}
\section{Problem Description}
\begin{tcolorbox}
	$2^{15} = 32768$ and the sum of its digits is $3 + 2 + 7 + 6 + 8 = 26$.
	What is the sum of the digits of the number $2^{1000}$?
\end{tcolorbox}

\section{Class Diagram}
\begin{center}
	\begin{tikzpicture}
		\umlclass{sint}
		{-sign: char\\
		-number: std::string}
		{
		+sint()\\
		+sint(std::string strnum)\\
		+sint(const char* strnum)\\
		+sint(long long n)\\
		+sint(int n)\\
		friend std::ostream\& operator<<(std::ostream\& os, const sint\& s)\\
		{\color{gray!30} \hrulefill}
		\\
		friend bool operator==(const sint\& lhs, const\& rhs)\\
		friend bool operator!=(const sint\& lhs, const\& rhs)\\
		friend bool operator<(const sint\& lhs, const\& rhs)\\
		friend bool operator>(const sint\& lhs, const\& rhs)\\
		friend bool operator<=(const sint\& lhs, const\& rhs)\\
		friend bool operator>=(const sint\& lhs, const\& rhs)\\
		{\color{gray!30} \hrulefill}
		\\
		sint\& operator-()\\
		friend bool operator+(const sint\& lhs, const\& rhs)\\
		friend bool operator-(const sint\& lhs, const\& rhs)\\
		sint\& operator++()\\
		sint operator++(int)\\
		sint\& operator- -()\\
		sint operator- -(int)\\
		sint\& operator+=(const sint\& other)\\
		sint\& operator-=(const sint\& other)\\
		friend sint operator*(const sint\& lhs, const\& rhs)\\
		friend sint operator/(const sint\& lhs, const\& rhs)\\
		sint\& operator*=(const sint\& other)\\
		sint\& operator/=(const sint\& other)\\
		friend sint operator\%(const sint\& lhs, const sint\& rhs)\\
		sint power(const sint\& n)\\
		sint digit\_sum()\\
		}
	\end{tikzpicture}
\end{center}

\section{Multiply}
\begin{center}
	\opmul{99}{524567}
\end{center}

\section{Division}
\begin{center}
	\intlongdivision{25789}{32}
\end{center}
